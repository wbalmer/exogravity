\section{An example: extracting the astrometry of a binary from a swap observation}
\label{sec:swap_example}

\subsection{Observations}
For this example, we are going to use some data obtained during an observation of HD~25535, on November, 11, 2019. The data are located in the folder \verb|HD25535/|.

The observation was carried out in \emph{dual-field} (one fiber used for the Fringe Tracker, and one for the ``science'' target), \emph{off-axis} (the roof mirror is used, and neither the FT fiber, not the science fiver, is centered in the field), using the ``swap'' template. During a swap observation, exposures are acquired sequentially with two configurations (see Figure~\ref{fig:swap}): 1- the FT fiber is centered on the target star, and the science fiber is the requested offset position; 2- the science fiber is one the target star, and the FT fiber at the given offset position compared to the science fiber (the two fibers are ``swapped'').

Note that this observation pattern can only be used for binary systems in which the two components are bright enough to feed the FT.

In the present case, a total of 8 exposures were aquired: 2 in the standard configure (FT on target), 2 in the swapped configuration (FT at offset position), then again 2 in standard configuration, and 2 in swapped configuration.

\begin{figure}
  \begin{center}
    %    \includegraphics[width=0.8\linewidth]{figures/swap.png}
    \caption{Illustration of the swap sequence}
  \end{center}
\end{figure}

The objective of the data reduction is to extract the astrometry of the binary. This is done by fitting the visibility of the science object. But to get a proper astrometry, several problems need to be taken care of:
\begin{enumerate}
\item{We are working with astrored files, so we need to take care of the FT zero point, dispersion, and metrology corrections}
\item{We are in dual-field, so we need to calculate the phase reference and to subtract it the observed visibilities}
\item{We are in swap mode, so remind that half the data are measuring $(\Delta\mathrm{RA},\Delta{}\mathrm{DEC})$, and half are measuring $(-\Delta\mathrm{RA},-\Delta{}\mathrm{DEC})$}
\end{enumerate}

\subsection{Imports}

For this section, we are going to use the \verb|cleanGravity| package, as well as the plotting functions contained in \verb|gravityPlots| (not automatically loaded), and a few standard packages.
If you want to run the examples, I suggest you import the following, and also define the \verb|DATADIR| variable to point to the HD~25535 directory.
\lstinputlisting[linerange={1-6}]{../python/swap_example.py}


\subsection{Load the data and apply the corrections}

\subsubsection{FT corrected visibilities}

Using \verb|cleanGravity|, the data can be loaded as \verb|GravityDualfieldAstrored| objects. You need to specify the path to the FITS file, and the FITS extension:
\lstinputlisting[linerange={8-9}]{../python/swap_example.py}

The attribute \verb|oi.visOi| is a \verb|VisOi| object, which contains the visibility data. The raw data can be accessed in \verb|oi.visOi.visData|. The package always takes care of correcting for the FT zero point, and the visibilities referenced to the FT are in \verb|oi.visOi.visRef|. You can easily visualize the difference between the two using the gravityPlot functions:
\lstinputlisting[linerange={11-14}]{../python/swap_example.py}

You can easily see how the FT tracking reference is shifting at each DIT by plotting the DITs individually:
\lstinputlisting[linerange={16-20}]{../python/swap_example.py}

\noindent{}Replace \verb|visData| with \verb|visRef|, and you should see that the FT corrected visibilities are now better aligned:
\lstinputlisting[linerange={22-26}]{../python/swap_example.py}


\subsubsection{OPD dispersion correction}

When working with fully reduced data (``scivis'' files), this step is taken care of by the GRAVITY pipeline. But when working with the ``astrored'' data, this correction is not applied by the pipeline. However, the OPD that needs to be added to the visibilities is calculated by the pipeline, and is available in \verb|oi.visOi.opdDisp|. So the correction is a simple one-liner:
\lstinputlisting[linerange={28-29}]{../python/swap_example.py}

\noindent{}For convenienence, this correction can also be applied directly when loading the data with \verb|cleanGravity|, simply by setting a \verb|corrDisp| keyword to \verb|"drs"| instead of the default \verb|"none"|:
\lstinputlisting[linerange={31-32}]{../python/swap_example.py}

Note that there exists an alternative calculation of the OPD dispersion correction, made by Sylvestre. This correction is applied in the same way, except that the \verb|opdDisp| values in this case are located in another OI of the FITS file (called SYLVESTRE). For the moment, this OI is not available by default in the pipeline reduced data. If you happen to have a file in which this OI is available\footnote{Usually, it means that you got it from Sylvestre himself}, you can request this alternative correction by setting \verb|corrDisp = "sylvestre"|:
\lstinputlisting[linerange={34-35}]{../python/swap_example.py}


\subsubsection{Metrology correction}

The metrology correction is also done by the pipeline on the ``scivis'' file, but not on the ``astrored'' files. Since we are working with the latter, we need to apply it manually. Again, this can be done by setting a keyword \verb|corrMet = "drs"| when loading the data:
\lstinputlisting[linerange={37-38}]{../python/swap_example.py}

But the sake of explanations, we are going to do it manually. The metrology correction comes from telescope-based measurements. This means that the required quantities must be retrieved from the flux OI, not from the visibility OI. They are available in the \verb|oi.fluxOi| object, as \verb|oi.visOi.telFcCorr| and \verb|oi.visOi.fcCorr|. The sum of the two give an OPD correction for each telescope. To convert it to a correction for each baseline, it is necessary to calculate the difference between the proper telescopes (two telescope per baseline).

In the \verb|visOi| object, the visibilities are ordered in decreasing order: $T_4$-$T_3$, $T_4$-$T_2$, $T_4$-$T_1$, $T_3$-$T_2$, $T_3$-$T_1$, $T_2$-$T_1$. The same if true for the telescopes in the \verb|fluxOi|: $T_4$, $T_3$, $T_2$, $T_1$. Thus, the conversion from telescope-based errors to baseline-based errors can be done using the following matrix:
\lstinputlisting[linerange={40-46}]{../python/swap_example.py}

With the matrix, the correction proceeds easily. You just have to remind that the values give an OPD correction, which needs to be divided by $\lambda$ to get a phase correction:
\lstinputlisting[linerange={48-50}]{../python/swap_example.py}

Again, there exists an alternate correction, which uses the SYLVESTRE OI, and which can be requested by setting \verb|corrMet = "sylvestre"| instead of \verb|"drs"|.

\subsection{Optional: calculate the mean of the DITs}

This step is optional. It is usually better to keep the DITs separated, but depending on how many files you have to reduced, and what the spectral resolution is, the computation can then be lengthy. Thus, when developing, it can be useful to average the DITs to speed up things.

Due to the rotation of the sky, averaging the visibilities is not as trivial as it sounds. The OPD on each baseline is changing with time, and thus the phase of the visibility is drifting. As a consequence, to average the visibilities properly, you first need to ``shift'' them to $\mathrm{OPD} = 0$. Ideally, this should be done using the proper astrometry of the target. Since it is unknown, the best way is to proceed using the position of the fiber as a substitue. To do that, you can use the \verb|recenterPhase| method of the \verb|visOi| object, to ``recenter'' the visbilities on the fiber position (available in RA/DEC in \verb|oi.sObjX| and \verb|oi.sObjY|). Then you can average the visibilities, the baselines UV coordinates, and shift the visibilities back to their original position:
\lstinputlisting[linerange={52-63}]{../python/swap_example.py}

Note that in the above code, the use of the \verb|np.tile| functions guarantees that the visibility data keep the same format (3 dimensional arrays, first dimension for the DIT number, second for the channel, and third for the walength).

\subsection{Separate and average the swap positions and calculate the phase reference}

At this point, you have the visibilities referenced to the FT, properly corrected for the dispersion, and for the metrology (i.e. for non common-path errors). Potentially, you have also averaged those visibilities. But these visibilities are still corrupted by the unknown metrology zero point (i.e. the phase picked up at the injection of the metrology laser in the system). Fortunately, this phase error is extremely stable. This means that, as long as you did not switched the observing mode (on-axis/off-axis) between the different exposures, this phase offset will be the same in all the data.

\noindent{}Consequently, for a swap observation, the phase at the two positions are respectively given by:
\begin{align*}
\text{Before the swap:} \qquad & \phi(u, v) = \frac{2\pi}{\lambda}\left(\Delta\mathrm{RA}\times{}u+\Delta\mathrm{DEC}\times{}v\right) + \phi_\mathrm{ref} \\
\text{After the swap:}  \qquad & \phi_\mathrm{swap}(u, v) = \frac{2\pi}{\lambda}\left(-\Delta\mathrm{RA}\times{}u-\Delta\mathrm{DEC}\times{}v\right) + \phi_\mathrm{ref}  
\end{align*}

\noindent{}Calculating the phase reference is thus just a matter of summing the phase of the two positions:
\begin{equation*}
  \phi_\mathrm{ref} = \frac{\phi + \phi_\mathrm{swap}}{2}
\end{equation*}

\noindent{}So the sequence to retrieve the phase reference is the following:
\begin{enumerate}
\item{Load all the files, and correct the visibilities referenced to the FT for dispersion and metrology}
\item{Separate the OIs in two groups: standard and swapped. This can be done using the \verb|oi.swap| boolean}
\item{Average the visibilities over each of the two groups, and extract the phase\footnote{The instrument measures the real and imaginary parts of the visibilities, not the phase. So you should always average the visibilities first and calculate the phase after! Never do the opposite (calculate the phase and then average).}}
\item{Calculate the phase reference}
\item{Subtract this phase reference to the visibilities of each file} 
\end{enumerate}
  
\noindent{}In Python, this goes as follows. First, all files are loaded and separated in two groups with:
\lstinputlisting[linerange={66-75}]{../python/swap_example.py}

\noindent{}Then, the average on each group can be calculated, again by first shifting the visibilities to 0 OPD:
\lstinputlisting[linerange={77-87}]{../python/swap_example.py}

\noindent{}The phase reference is extarcted and removed from all the OIs:
\lstinputlisting[linerange={89-92}]{../python/swap_example.py}



\subsection{Fit for the astrometry}

From there on, the visibility you have in \verb|oi.visOi.visRef| should correspond to the astrophysical visibilities. The last thing to do is to fit for the astrometry of the binary.

A simple model for the visibilities, with two free parameters $\Delta\mathrm{RA}$ and $\Delta\mathrm{DEC}$ can be built from the following equation, which gives the visibility as a function of the baseline coordinates:
\begin{equation*}
  V_{(\Delta{}\mathrm{RA}, \Delta{}\mathrm{DEC})}(U, V) = |V|\exp\left\{{i\,2\pi}\left(\Delta\mathrm{RA}\times{}U+\Delta\mathrm{DEC}\times{}V\right)\right\}
\end{equation*}
\noindent{}In which $|V|$ is a visibility modulus.

A basic fit of this model could be a linear fit, in which a scaling factor is adjusted for each baseline and each DIT, to take into account the variations in the transmission. It should be noted, though, that you should never let the scaling factor be negative. A negative scaling factor can ``revert'' the fringes, making a completely out-of-phase model (i.e. with a phase offset of $\pi$ compared to the data) look ``good''. One simple strategy is to operate a linear fit, and if the resulting scaling coefficient is negative, set it to 0. Alternatively, if the model include other elements, for example an underlying polynomial to model a contamination on the detector, you could set the wavelet (the visibility model) to 0, and redo the fit only with the polynomial.

The corresponding code would be something along this line:
\lstinputlisting[linerange={95-123}]{../python/swap_example.py}

\noindent{}Of course, this is valid for the standard position. For the swapped position, the measured astrometry is the opposite of the true astrometry of the binary:
\begin{equation*}
  V^{\mathrm{swap}}_{(\Delta{}\mathrm{RA}, \Delta{}\mathrm{DEC})}(U, V) = |V|\exp\left\{{-i\,2\pi}\left(\Delta\mathrm{RA}\times{}U+\Delta\mathrm{DEC}\times{}V\right)\right\}
\end{equation*}

From this, you can calculate a $\chi^2$ map for each OI, and extract the correponding best astrometry, with its error bars. You have reduced a GRAVITY dual-field and off-axis data set!

This example is pretty much what is contained in the \verb|swapReduce.py| script of the \verb|exoGravity| package.

\subsection{Optional: OPD-based calculation of the $\chi^2$ map}

There is another way to calculate the astrometry $\chi^2$ maps, which is used in the exoGravity pipeline, because it goes faster and provides similar results.

The basic idea is that, for each baseline, the RA and DEC values are combined as a sum, virtually acting as a single parameter. This means that what we are really trying to fit in the data is the OPD at each baseline. Thus, you cal calculate $\chi^2$ maps in OPD (one per baseline and per file), simply by fiting a single parameter model like:
\begin{equation*}
  V_{\mathrm{OPD}}(U, V) = |V|\exp\left\{i\,\frac{2\pi}{\lambda}\mathrm{OPD}\right\}
\end{equation*}

The $(\Delta\mathrm{RA}, \Delta\mathrm{DEC})$, can then be calculated from these OPD maps by simple summation, i.e. for each $\Delta\mathrm{RA}, \Delta\mathrm{DEC}$ values, and each baseline $(u, v)$, you can calculate the corresponding OPD:
\begin{equation*}
  \mathrm{OPD}_{\Delta\mathrm{RA}, \Delta\mathrm{DEC}}(u, v) = \Delta\mathrm{RA}\times{}u+\Delta\mathrm{DEC}\times{}v
\end{equation*}

\noindent{}And then sum:
\begin{equation*}
  \chi^2(\Delta{}\mathrm{RA}, \Delta{}\mathrm{DEC}) = \sum_{u, v} \chi^2\left(\mathrm{OPD}_{\Delta\mathrm{RA}, \Delta\mathrm{DEC}}(u, v)\right)
\end{equation*}

\noindent{}The corresponding code is left as an exercice to the reader.
