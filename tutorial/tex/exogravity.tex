\section{The exoGravity package}

\subsection{Installation}

\subsubsection{Requirements}

\begin{itemize}
\item{Tested on Linux/Debian 8}
\item{Python 2.7 or Python 3.x, ideally with x>=5 (required for ruamel.yaml)}
\item{Astropy (>=2.0.2)}
\item{ruamel.yaml >=0.16.5 (recommended) or alternatively pyyaml >=3.11}
\item{NumPy (>=1.16.5), Scipy (>=0.19.0)}
\item{cleanGravity (see installation procedure below)}  
\end{itemize}


\subsubsection{Installation procedure}
To be improved

Download cleanGravity, add to path, then download exoGravity, makes scripts executable (\verb|chmod +x|), add to path. Test it. You're done!



\subsection{Usage}

The exoGravity package provides 3 scripts, which should be used sequentially to reduce the data. The scripts are the following:
\begin{itemize}
\item{\verb|astrometry_reduce.py|: is used to extract the astrometry from the observations. The script will output the best astrometry on the terminal, and will add the astrometric solution for each \verb|planet_oi| in the YAML configuration file.} 
\item{\verb|spectrum_reduce.py|: is used to extract the spectrum from the observations. The script will create a ``contrast.txt'' and ``covariance.txt'', to store the final spectrum and its error bars. Ideally, an astrometric solution should be provided for each planet observation in the YAML configuration file. If not, the script will assume the fiber to be perfectly centered on the planet, which may lead to bad results.}
\item{\verb|swap_reduce.py| is only used for off-axis observations, for which the astrometry of the reference binary is not known with enough precision. This script will extract he astrometric solution for the binary from the observations themselves.}
\end{itemize}

These scripts use a common YAML configuration file, which describes the data files, the reduction parameters, etc. The exoGravity package comes with a tool to help you create a properly formatted YAML file: \verb|create_config.py|. This script takes at least one argument, which is the path to the directory in which the astrored files are stored. It goes through the files, identifies on-planet, on-star, and swap observations, and created an associated config file. lease see the documentation for further options.

To use these scripts, start by putting all the ``astrored'' files corresponding to your observation (i.e. on-planet, on-star, possibly swap reference) in a common directory. To test the scripts, you can use one of the directory contained in the \verb|examples| directory of this tutorial.

Instead of creating a configuration file from scratch yourself, it is strongly advised to use the script provided, and to tweak the parameters afterwards. Run:
\begin{lstlisting}[language=bash]
create_config.py datadir=./examples/betaPictoris_2018/ output=betaPictoris2018.yml
\end{lstlisting}

This will create the YAML file. You can now open it with a text editor. For this demo, we will change a few parameters to accelerate the reduction. First, in the \verb|general| section, look for the \verb|gofast| parameter, and set it to \verb|true|. Then, search for the number of points in the RA/DEC maps (\verb|n_ra| and \verb|n_dec|, and set them to 50 instead of the default 100). Note that these changes could also have been requested at the creation of the file, by calling:
\begin{lstlisting}[language=bash]
create_config.py datadir=./examples/betaPictoris_2018/ output=betaPictoris2018.yml gofast=True nra=50 ndec=50
\end{lstlisting}

\noindent{}The last change we will make is to skip the reduction of some of the files. The YAML format supports commenting (with the char \verb|#|). In the \verb|general| section of the configuration file, look for the \verb|reduce| keyword. This gives a list (one element per line, preceded by a dash, as per YAML specification) of keys to the planet files that will be reduced. The keys refer to the elements of the \verb|planet_ois| section of this same configuration file. For this example, we will only keep the first 5 files.

\noindent{}The \verb|general| section of you config file should now look something like:
\begin{verbatim}
general:
  contrast_file: null
  datadir: /data1/gravity/BetaPictoris/2018-09-22/reduced/
  declim:
  - 117.47058823529412
  - 127.47058823529412
  gofast: true
  n_dec: 50
  n_opd: 100
  n_ra:50
  noinv: false
  phaseref_mode: DF_STAR
  ralim:
  - 65.0
  - 75.0
  reduce:
  - p0
  - p1
  - p2
  - p3
  - p4
  - p5
#  - p6
#  - p7
#  - p8
#  - p9
#  - p10
#  - p11
#  - p12
#  - p13
#  - p14
#  - p15
#  - p16
  save_fig: true
  star_diameter: 0.0
  star_order: 3
\end{verbatim}

\noindent{Now, you can extract the astrometry using:}
\begin{lstlisting}[language=bash]
astrometry_reduce.py config_file=betaPictoris2018.yml
\end{lstlisting}
\noindent{}At the end, the script will output the RA/DEC solution, in mas: (RA=68.27, DEC=126.45). The script will also have added the astrometric solution for each of the planet files in the configuration file. If you reopen the file in a text editor, you should see the in the \verb|planet_ois| section, each element which was not commented in the \verb|reduce| list now has an \verb|astrometric_solution|. Note that if you are using exoGravity with the pyyaml library for managing the YAML configuration files (see installation section), the lines commented in the YAML file do not survive read/write cycles. Thus, is using pyyaml, the \verb|reduce| list in \verb|general| section is now shorter.

Now that each planet files has an astrometric solution, you can move to the spectrum extaction:
\begin{lstlisting}[language=bash]
spectrum_reduce.py config_file=betaPictoris2018.yml outputdir=./result
\end{lstlisting}

\noindent{}When the script is finished, you will have two files in the ``result'' dir. You can plot the spectrum with any tool you want. For example:
\begin{lstlisting}[language=python]
  import matplotlib.pyplot as plt
  import numpy as np
  data = np.loadtxt("./result/contrast.txt", skiprows=2)
  plt.figure()
  plt.plot(data[:, 0], data[:, 1])
  plt.title("Contrast spectrum for beta Pictoris b")
  plt.xlabel("Wavelength (um)")
  plt.ylabel("Contrast")  
  plt.show()
\end{lstlisting}

\subsection{Configuration File}

The reduction done by the exoGravity package is parameterized using a YAML configuration file. This configuration file contains 3 or 4 main levels, depending on the type of observation, defined below.

\subsubsection{General}
The \verb|general| section contains parameters which are not file-specific, and which are used throughout the entire reduction.

To be written






